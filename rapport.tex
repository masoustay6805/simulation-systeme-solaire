\newif\ifCOR
%\CORtrue

\newif\ifIncludesEmbedded
\IncludesEmbeddedtrue

% Preamble
\documentclass[a4paper,10pt]{article}

% Packages
\usepackage[francais]{babel} % Typographie
\usepackage[T1]{fontenc}     % Saisie en
\usepackage[utf8]{inputenc} % français
\usepackage{arev}
\usepackage[usenames,dvipsnames,svgnames,table]{xcolor}

% Réglages généraux
\usepackage[left=3.0cm, right=2.3cm, top=3cm, bottom=2.5cm]{geometry} % taille de la feuille
\usepackage{fancyhdr} % Titre courant
\usepackage{setspace} % Interligne
\usepackage{lscape}   % Mode paysage
\usepackage{multicol}  % Plusieurs colonnes
\usepackage{placeins}
\usepackage{caption}

% Packages pour les tableaux
\usepackage{array}     % Outils supplémentaires
\usepackage{multirow}  % Colonnes multiples
\usepackage{tabularx}  % Largeur totale donnée
\usepackage{longtable} % sur plusieurs pages
\usepackage{fancybox}

% paquets mathématiques
\usepackage{amsmath, amssymb, mathrsfs, theorem}
\usepackage{wasysym} % symboles exotiques : \smiley...

\usepackage{lastpage} % pour avoir \pageref{LastPage} : le nombre total de pages du doc
\usepackage{verbatim} % for \begin{comment}
\usepackage{fancyvrb}

% Color scheme for listings
\usepackage{textcomp}
\definecolor{listinggray}{gray}{0.9}
\definecolor{lbcolor}{rgb}{0.9,0.9,0.9}

% Listings configuration
\usepackage{listings}
\lstset{
    tabsize=4,
    rulecolor=,
    language=python,
    basicstyle=\scriptsize,
    upquote=true,
    aboveskip={1.5\baselineskip},
    columns=fixed,
    numbers=left,
    showstringspaces=false,
    extendedchars=true,
    breaklines=true,
    prebreak = \raisebox{0ex}[0ex][0ex]{\ensuremath{\hookleftarrow}},
    frame=shadowbox,
    rulesepcolor=\color{black},
    showtabs=false,
    showspaces=false,
    showstringspaces=false,
    identifierstyle=\ttfamily,
    keywordstyle=\color[rgb]{0,0,1},
    commentstyle=\color{red},
    stringstyle=\color[rgb]{0.627,0.126,0.941},
    literate={á}{{\'a}}1 {ã}{{\~a}}1 {é}{{\'e}}1 {è}{{\`e}}1 {à}{{\`a}}1,
}

% Other
\usepackage{hyperref}
\usepackage{float}
\usepackage{graphicx}
\graphicspath{{../images/}}
\hypersetup{colorlinks=false}
\hypersetup{pdfauthor=STAHLI Jules, pdfkeywords=stage}
\hypersetup{pdfstartpage=1}
\hypersetup{pdfpagemode=None} %FullScreen, None
\hypersetup{pdfpagelayout=SinglePage} %SinglePage, OneColumn, TwoColumnLeft, TwoColumnRight

\hypersetup{pdfstartview=Fit} %Fit, FitH, FitV, FitB, FitBH, FitBV

% Commands
\newcommand{\doctitle}{Rapport}
\newcommand{\docsubtitle}{Simulation système solaire}
\newcommand{\docauthor}{David Rajohnson, Jules Stähli}
\newcommand{\docversion}{v2.0}
\newcommand{\filename}{rapport.tex}
\newcommand{\docdate}{21 décembre 2021}

% Document info
\title{\doctitle}
\author{\docauthor}
\date{\docdate}

% Header and Footer
\pagestyle{fancy}
\fancyhf{}
\rhead{\docauthor}
\lhead{\doctitle~-~\docsubtitle}
\rfoot{Page \thepage~sur~\pageref{LastPage}}
\lfoot{\docdate~-~\docversion}

\setlength\parindent{0pt} %noindent for all the document

% Document
\begin{document}
    \begin{center}
    {\huge \doctitle~-~\docsubtitle}\newline
    \end{center}

    \tableofcontents

    \newpage

    \section{Introduction}

    Pour réaliser une simulation d'un système planétaire, plusieurs facteurs sont à prendre en compte.
    Puisque dans notre univers, tous les corps sont soumis à des forces.
    Nous allons donc utiliser pour cela les trois lois de Newton:

    \begin{enumerate}
        \item Si un corps est immobile (ou en mouvement rectiligne uniforme), alors la somme des forces qu’il subit, appelée force résultante, est nulle. ($\vec{F} = 0$).
        \item La force résultante subit par un corps est égale à la masse de ce dernier multipliée par son accélération. ($\vec{F} = m * a$).
        \item Si un corps A subit une force de la part d’un corps B, alors le corps B subit une force de réaction de sens opposé et de même intensité ($\vec{F}_{b\to{a}} = -\vec{F}_{a\to{b}}$).
    \end{enumerate}
    
    \subsection{Réalisation de la simulation d'un système planétaire}\label{subsec:réalisation-de-la-simulation-d'un-système-planétaire}

    \subsubsection{Calcul de la vélocité d'une planète}

    $\vec{v}_{p}(0) = \sqrt{\frac{GM * (1 + e_{p})}{a_{p}(1 - e_{p})} * \frac{\vec{r}_{p}}{\|\vec{r}_{p}\|}}$

    \begin{lstlisting}[language=c,label={lst:lstlisting3}]
        vec2 compute_velocity(system_t *system, planet_t planet) {
               return vec2_mul(
           sqrt((G * system->star.mass * (1 + planet.eccentricity)) / (planet.semi_major_axis * (1 - planet.eccentricity))),
           vec2_normalize(vec2_create(-planet.pos.y, -planet.pos.x))
           );
       }
    \end{lstlisting}
    
    \subsubsection{Calcul de l'accélération d'une planète}

    $\vec{F}_{B\to{}A} = G * \frac{m_{A}m_{B}}{\|\vec{r}_{AB}\|^{3}} * \vec{r}_{AB}$

    \begin{lstlisting}[language=c,label={lst:lstlisting4}]
        vec2 compute_acceleration(system_t *system, planet_t planet, int32_t planet_index) {
    vec2 f_res = vec2_create_zero();
    for (int32_t i = -1; i < system->nb_planets; i++) {
        if (i != planet_index) {
            planet_t planet_b = i == -1 ? system->star : system->planets[i];
            vec2 rab = vec2_sub(planet_b.pos, planet.pos);
            vec2 fab = vec2_mul(G * ((planet.mass * planet_b.mass) / pow(vec2_norm(rab), 3)), rab);
            f_res = vec2_add(f_res, fab);
        }
    }
    return vec2_mul(1 / planet.mass, f_res);
}
    \end{lstlisting}
    
    \subsubsection{Calcul de la position initial d'une planète}

    $\vec{x}_{p}(\Delta{}t) = \vec{x}_{p}(0) + \Delta{}\vec{tv}_{p}(0) + \frac{(\Delta{}t)^{2}}{2} * \vec{a}_{p}$

    \begin{lstlisting}[language=c,label={lst:lstlisting5}]
        vec2 compute_initial_position(system_t *system, planet_t planet, int32_t planet_index, double delta_t) {
    vec2 velocity = compute_velocity(system, planet);
    vec2 acceleration = compute_acceleration(system, planet, planet_index);
    return vec2_add(vec2_add(planet.pos, vec2_mul(delta_t, velocity)), vec2_mul(pow(delta_t, 2) / 2, acceleration));
    }
    \end{lstlisting}
    
    \subsubsection{Mise à jour de la position d'une planète}

    $\vec{x}_{p}(t + \Delta{}t) = \vec{x}_{p}(t) + \Delta{}\vec{tv}_{p}(t) + \frac{(\Delta{}t)^{2}}{2} * \vec{a}_{p}(t)$

    \begin{lstlisting}[language=c,label={lst:lstlisting6}]
        vec2 compute_next_position(system_t *system, planet_t planet, int32_t planet_index, double delta_t) {
    vec2 acceleration = compute_acceleration(system, planet, planet_index);
    return vec2_add(vec2_sub(vec2_mul(2, planet.pos), planet.prec_pos), vec2_mul(pow(delta_t, 2), acceleration));
    }
    \end{lstlisting}

    \subsubsection{Création d'une étoile(dans notre cas: le soleil)}

    Le soleil sera représenté comme une planète donc aura la structure d'une planète mais son comportement sera différentes des autres planètes.
    Pour ce faire, on a tout d'abord défini la structure d'une planète:

    \begin{lstlisting}[language=c,label={lst:lstlisting}]
        typedef struct _planet
        {
            double mass; //masse de la planète
            int32_t color; //couleur sur la simulation
            double radius; //Grandeur de la planète
            double semi_major_axis; //Distance par rapport au soleil
            double eccentricity; //excentricité
            vec2 pos;      // x(t)
            vec2 prec_pos; // x(t - dt)
        } planet_t;
    \end{lstlisting}

    Comme convenu, pour la création de notre étoile nous avons donc eu besoin d'implémenter une fonction qui permet de créer les planètes.
    Cette fonction contient également un champ qui permettra de préciser si cette planète créée est une étoile(\texttt{champ: is\char`_star}):
    
    \begin{lstlisting}[language=c,label={lst:lstlisting2}]
        planet_t create_planet(double mass, int color, double radius, double semi_major_axis, double eccentricity, bool is_star) {
        planet_t planet = { //attribution de tous les paramètres
            .mass = mass,
            .color = color,
            .radius = radius,
            .semi_major_axis = semi_major_axis,
            .eccentricity = eccentricity,
        };

        if (is_star) { //Si c'est une étoile on le place au milieu donc vecteur 0
            planet.prec_pos = planet.pos = vec2_create_zero();
        } else { //Sinon on la place par rapport à sa distance au soleil(étoile) et son excentricité
            planet.prec_pos = planet.pos = vec2_create(semi_major_axis * (1 - eccentricity), 0);
        } //La planète est aligné horizontalement au soleil
        return planet;
    }
    \end{lstlisting}
    
    \subsubsection{Ajout de planètes autour de étoile(Soleil}

    Ici on a déjà la base pour créer chaque planète, donc on désire en créer chaque planète autour de l'étoile.
    Pour cela nous avons décidé de créer les planètes à partir d'un fichier \emph{.csv} pour ne pas avoir à modifier tout le code à chaque création de planètes.
    Notre fichier \emph{.csv} est composé comme ci-dessous:

    \begin{figure}[H]
        \begin{tabular}{| l | l | l | l | l | l |}
            \hline
            \texttt{Name} & \texttt{Mass} & \texttt{Semi major axis} & \texttt{Eccentricity} & \texttt{Radius} & \texttt{Color} \\
            \hline\hline
            Sun & 1.989e30 & 0 & 0 & 8 & 0x00FFFF00 \\
            \hline
            Mercury & 3.285e23 & 57.909e9 & 0.2056 & 0.5 & 0x0096764B \\
            \hline
            Venus & 4.867e24 & 108.209e9 & 0.0067 & 2 & 0x00BC611C \\
            \hline
            Earth & 5.972e24 & 149.596e9 & 0.0167 & 1 & 0x003A57D0 \\
            \hline
            Mars & 6.417e23 & 227.923e9 & 0.0935 & 1.5 & 0x00B9351A \\
            \hline
            Jupiter & 1.989e30 & 378.570e9 & 0.0489 & 2.5 & 0x00C27841 \\
            \hline
            Saturn & 568.32e27 & 1434e9 & 0.0565 & 2.25 & 0x00A3946B \\
            \hline
        \end{tabular}
        \captionof{table}{Données~-~\emph{planetes.csv}}\label{fig:figure}
    \end{figure}

    Pour créer chaque planète à partir de ce tableau, on a créer une fonction qui permettra de récupérer chaque élément du tableau et l'attribuer dans le bon champs de la planète.

    \subsubsection{Lecture d'un fichier CSV}
\end{document}